%%%%%%%%%%%%%%%%%%%%%%%%%%%%%%%%%%%%%%%%%
% a0poster Landscape Poster
% LaTeX Template
% Version 1.0 (22/06/13)
%
% The a0poster class was created by:
% Gerlinde Kettl and Matthias Weiser (tex@kettl.de)
% 
% This template has been downloaded from:
% http://www.LaTeXTemplates.com
%
% License:
% CC BY-NC-SA 3.0 (http://creativecommons.org/licenses/by-nc-sa/3.0/)
%
%%%%%%%%%%%%%%%%%%%%%%%%%%%%%%%%%%%%%%%%%

%----------------------------------------------------------------------------------------
%	PACKAGES AND OTHER DOCUMENT CONFIGURATIONS
%----------------------------------------------------------------------------------------

\documentclass[a0,landscape]{a0poster}

\usepackage{multicol} % This is so we can have multiple columns of text side-by-side
\columnsep=100pt % This is the amount of white space between the columns in the poster
\columnseprule=3pt % This is the thickness of the black line between the columns in the poster

\usepackage[svgnames]{xcolor} % Specify colors by their 'svgnames', for a full list of all colors available see here: http://www.latextemplates.com/svgnames-colors

\usepackage{times} % Use the times font
%\usepackage{palatino} % Uncomment to use the Palatino font
\usepackage[english]{babel}% Use english package

\usepackage{graphicx} % Required for including images
\graphicspath{{figures/}} % Location of the graphics files
\usepackage{booktabs} % Top and bottom rules for table
\usepackage[font=small,labelfont=bf]{caption} % Required for specifying captions to tables and figures
\usepackage{amsfonts, amsmath, amsthm, amssymb} % For math fonts, symbols and environments
\usepackage{wrapfig} % Allows wrapping text around tables and figures

\begin{document}

%----------------------------------------------------------------------------------------
%	POSTER HEADER 
%----------------------------------------------------------------------------------------

% The header is divided into three boxes:
% The first is 55% wide and houses the title, subtitle, names and university/organization
% The second is 25% wide and houses contact information
% The third is 19% wide and houses a logo for your university/organization or a photo of you
% The widths of these boxes can be easily edited to accommodate your content as you see fit

\begin{minipage}[b]{0.80\linewidth}
\veryHuge \color{NavyBlue} \textbf{Medical diagnostics over video} \color{Black}\\ % Title
\Huge\textit{Extracting the pulse and oxygen levels in blood from video of a subject}\\[1cm] % Subtitle
\huge \textbf{D Deriso, A Fallou \& N Banerjee}\\ % Author(s)
%\huge University and Department Name\\ % University/organization
\end{minipage}
%
\begin{minipage}[b]{0.19\linewidth}
\includegraphics[width=20cm]{images/red_trough.png} % Logo or a photo of you, adjust its dimensions here
\end{minipage}

\vspace{1cm} % A bit of extra whitespace between the header and poster content

%----------------------------------------------------------------------------------------

\begin{multicols}{4} % This is how many columns your poster will be broken into, a poster with many figures may benefit from less columns whereas a text-heavy poster benefits from more

%----------------------------------------------------------------------------------------
%	ABSTRACT
%----------------------------------------------------------------------------------------

\color{Navy} % Navy color for the abstract

\begin{abstract}

In both developing and developed countries, reducing the cost of medical care is a primary goal of science and government.
 In this project we seek to find and extract information from a video of a human that tells us the pulse rate and the oxygen level saturation of the blood. 
 We therefore aim to create a virtual pulse oximeter: the ultimate non-invasive, equipment-free medical diagnostics tool, which could be deployed to anyone with video recording capabilities.
 Features were chosen to be related to the three color channel intensity values, with the idea that changing color of the video would relate to blood flow around the body.
 Specifically we chose to consider the largest amplitude fourier components of the time course of the data from each pixel.
 Extensive preprocessing was required on both the video data and the pulse oximeter data to enable training.
 INSERT MORE HERE WHEN WE HAVE RESULTS.

\end{abstract}

%----------------------------------------------------------------------------------------
%	INTRODUCTION
%----------------------------------------------------------------------------------------

\color{SaddleBrown} % SaddleBrown color for the introduction

\section*{Introduction}

Cardiovascular health is the \emph{sin qua non} of human life. Early detection of cardiovascular disease is of paramount importance in public health. This project aims to develop a method to visualize the perfusion of blood through the skin via pulse oximetry. Pulse oximetry is a technique that exploits the fact that oxygenated and deoxygenated hemoglobin changes the color of red blood cells. The technique maps these changes in rgb color of the visible skin to the invisible presence of oxygenated vs deoxygenated blood in the local vasculature underneath the skin.

Previous studies have shown that video obtained from an ordinary webcam can be used to visualize perfusion by selectively amplifying temporal frequencies in video {http://people.csail.mit.edu/mrub/vidmag/}. A study by the MIT CSAIL showed that this technique can also be used to infer heart rate from the person being taped. The present project aims to extend this work to detect the relative changes in oxygenated vs deoxygenated blood and reconstruct the pulse oximeter waveform from an ordinary webcam video.

%----------------------------------------------------------------------------------------
%	OBJECTIVES
%----------------------------------------------------------------------------------------

\color{DarkSlateGray} % DarkSlateGray color for the rest of the content

\section*{Main Objectives}

\begin{enumerate}
\item Extract pixel time course information from a video.
\item Extract pulse oximeter wave information from a pulse oximeter.
\item Build software that can simultaneously record pulse oximeter wave values and time values while recording from a video.
\item Estimate the number of training examples needed to implement linear regression using learning theory.
\item Implement linear regression and the train the weight matrix for the features from the video.
\item Test the newly learned weight matrix on further videos.

\end{enumerate}

%----------------------------------------------------------------------------------------
%	Feature selection and Pre-processing
%----------------------------------------------------------------------------------------

\section*{Feature selection and pre-processing}



%------------------------------------------------

\subsection*{Data processing}

\includegraphics{images/flow_chart_1.jpg}

\includegraphics{images/flow_chart_2.jpg}
Nulla vel nisl sed mauris auctor mollis non sed. 

\begin{equation}
E = mc^{2}
\label{eqn:Einstein}
\end{equation}

Curabitur mi sem, pulvinar quis aliquam rutrum. (1) edf (2)
, $\Omega=[-1,1]^3$, maecenas leo est, ornare at. $z=-1$ edf $z=1$ sed interdum felis dapibus sem. $x$ set $y$ ytruem. 
Turpis $j$ amet accumsan enim $y$-lacina; 
ref $k$-viverra nec porttitor $x$-lacina. 

Vestibulum ac diam a odio tempus congue. Vivamus id enim nisi:

\begin{eqnarray}
\cos\bar{\phi}_k Q_{j,k+1,t} + Q_{j,k+1,x}+\frac{\sin^2\bar{\phi}_k}{T\cos\bar{\phi}_k} Q_{j,k+1} &=&\nonumber\\ 
-\cos\phi_k Q_{j,k,t} + Q_{j,k,x}-\frac{\sin^2\phi_k}{T\cos\phi_k} Q_{j,k}\label{edgek}
\end{eqnarray}
and
\begin{eqnarray}
\cos\bar{\phi}_j Q_{j+1,k,t} + Q_{j+1,k,y}+\frac{\sin^2\bar{\phi}_j}{T\cos\bar{\phi}_j} Q_{j+1,k}&=&\nonumber \\
-\cos\phi_j Q_{j,k,t} + Q_{j,k,y}-\frac{\sin^2\phi_j}{T\cos\phi_j} Q_{j,k}.\label{edgej}
\end{eqnarray} 

Nulla sed arcu arcu. Duis et ante gravida orci venenatis tincidunt. Fusce vitae lacinia metus. Pellentesque habitant morbi. $\mathbf{A}\underline{\xi}=\underline{\beta}$ Vim $\underline{\xi}$ enum nidi $3(P+2)^{2}$ lacina. Id feugain $\mathbf{A}$ nun quis; magno. Fusce convallis rutrum turpis, quis aliquet enim accumsan id. Vestibulum ullamcorper porttitor convallis. Integer sagittis interdum malesuada. Class aptent taciti sociosqu ad litora torquent per conubia nostra, per inceptos himenaeos. Sed adipiscing tristique orci at ullamcorper. Morbi accumsan, urna et porttitor pulvinar, lacus risus dignissim massa. Proin sollicitudin. Pellentesque eget orci eros. Fusce ultricies, tellus et pellentesque fringilla, ante massa luctus libero, quis tristique purus urna nec nibh.

%----------------------------------------------------------------------------------------
%	RESULTS 
%----------------------------------------------------------------------------------------

\section*{Results}

Donec faucibus purus at tortor egestas eu fermentum dolor facilisis. Maecenas tempor dui eu neque fringilla rutrum. Mauris \emph{lobortis} nisl accumsan. Aenean vitae risus ante. Pellentesque condimentum dui. Etiam sagittis purus non tellus tempor volutpat. Donec et dui non massa tristique adipiscing.
%
\begin{wraptable}{l}{12cm} % Left or right alignment is specified in the first bracket, the width of the table is in the second
\begin{tabular}{l l l}
\toprule
\textbf{Treatments} & \textbf{Response 1} & \textbf{Response 2}\\
\midrule
Treatment 1 & 0.0003262 & 0.562 \\
Treatment 2 & 0.0015681 & 0.910 \\
Treatment 3 & 0.0009271 & 0.296 \\
\bottomrule
\end{tabular}
\captionof{table}{\color{Green} Table caption}
\end{wraptable}
%
Phasellus imperdiet, tortor vitae congue bibendum, felis enim sagittis lorem, et volutpat ante orci sagittis mi. Morbi rutrum laoreet semper. Morbi accumsan enim nec tortor consectetur non commodo nisi sollicitudin. Proin sollicitudin. Pellentesque eget orci eros. Fusce ultricies, tellus et pellentesque fringilla, ante massa luctus libero, quis tristique purus urna nec nibh.

Nulla ut porttitor enim. Suspendisse venenatis dui eget eros gravida tempor. Mauris feugiat elit et augue placerat ultrices. Morbi accumsan enim nec tortor consectetur non commodo. Pellentesque condimentum dui. Etiam sagittis purus non tellus tempor volutpat. Donec et dui non massa tristique adipiscing. Quisque vestibulum eros eu. Phasellus imperdiet, tortor vitae congue bibendum, felis enim sagittis lorem, et volutpat ante orci sagittis mi. Morbi rutrum laoreet semper. Morbi accumsan enim nec tortor consectetur non commodo nisi sollicitudin.

\begin{center}\vspace{1cm}
\includegraphics[width=0.8\linewidth]{images/red_trough.png}
\captionof{figure}{\color{Green} Figure caption}
\end{center}\vspace{1cm}

In hac habitasse platea dictumst. Etiam placerat, risus ac.

Adipiscing lectus in magna blandit:

\begin{center}\vspace{1cm}
\begin{tabular}{l l l l}
\toprule
\textbf{Treatments} & \textbf{Response 1} & \textbf{Response 2} \\
\midrule
Treatment 1 & 0.0003262 & 0.562 \\
Treatment 2 & 0.0015681 & 0.910 \\
Treatment 3 & 0.0009271 & 0.296 \\
\bottomrule
\end{tabular}
\captionof{table}{\color{Green} Table caption}
\end{center}\vspace{1cm}

Vivamus sed nibh ac metus tristique tristique a vitae ante. Sed lobortis mi ut arcu fringilla et adipiscing ligula rutrum. Aenean turpis velit, placerat eget tincidunt nec, ornare in nisl. In placerat.

\begin{center}\vspace{1cm}
\includegraphics[width=0.8\linewidth]{images/red_trough.png}
\captionof{figure}{\color{Green} Figure caption}---
\end{center}\vspace{1cm}

%----------------------------------------------------------------------------------------
%	CONCLUSIONS
%----------------------------------------------------------------------------------------

\color{SaddleBrown} % SaddleBrown color for the conclusions to make them stand out

\section*{Conclusions}

\begin{itemize}
\item Pellentesque eget orci eros. Fusce ultricies, tellus et pellentesque fringilla, ante massa luctus libero, quis tristique purus urna nec nibh. Phasellus fermentum rutrum elementum. Nam quis justo lectus.
\item Vestibulum sem ante, hendrerit a gravida ac, blandit quis magna.
\item Donec sem metus, facilisis at condimentum eget, vehicula ut massa. Morbi consequat, diam sed convallis tincidunt, arcu nunc.
\item Nunc at convallis urna. isus ante. Pellentesque condimentum dui. Etiam sagittis purus non tellus tempor volutpat. Donec et dui non massa tristique adipiscing.
\end{itemize}

\color{DarkSlateGray} % Set the color back to DarkSlateGray for the rest of the content

%----------------------------------------------------------------------------------------
%	FORTHCOMING RESEARCH
%----------------------------------------------------------------------------------------

\section*{Forthcoming Research}

Vivamus molestie, risus tempor vehicula mattis, libero arcu volutpat purus, sed blandit sem nibh eget turpis. Maecenas rutrum dui blandit lorem vulputate gravida. Praesent venenatis mi vel lorem tempor at varius diam sagittis. Nam eu leo id turpis interdum luctus a sed augue. Nam tellus.

 %----------------------------------------------------------------------------------------
%	REFERENCES
%----------------------------------------------------------------------------------------

\nocite{*} % Print all references regardless of whether they were cited in the poster or not
\bibliographystyle{plain} % Plain referencing style
\bibliography{sample} % Use the example bibliography file sample.bib

%----------------------------------------------------------------------------------------
%	ACKNOWLEDGEMENTS
%----------------------------------------------------------------------------------------

\section*{Acknowledgements}

Etiam fermentum, arcu ut gravida fringilla, dolor arcu laoreet justo, ut imperdiet urna arcu a arcu. Donec nec ante a dui tempus consectetur. Cras nisi turpis, dapibus sit amet mattis sed, laoreet.

%----------------------------------------------------------------------------------------

\end{multicols}
\end{document}