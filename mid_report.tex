\documentclass[12pt]{article}
%Had an issue with too many packages so needed to put this in to work around it.
\usepackage{etex}

\usepackage{amssymb,amsmath,amsthm,mathtools}
\usepackage{hyperref,cleveref}
\usepackage[margin=1.25in]{geometry}
\usepackage{graphicx,ctable,booktabs} 
\usepackage[parfill]{parskip} % begin paragraphs on empty line rather than indent
\usepackage{fancybox}
\usepackage{tipa} % for \textpipe
\usepackage{caption}
\usepackage{bbm}


\usepackage{tikz}
\usetikzlibrary{shapes,arrows,positioning}

\usepackage{epstopdf} % eps to pdf, declare graphics
\usepackage{soul} % enable highlighting text: use \hl{your text here}
\DeclareGraphicsRule{.tif}{png}{.png}{`convert #1 `dirname #1`/`basename #1 .tif`.png}
\def\thesection{\arabic{section}} % adjust section and subsection labelling 
\def\thesubsection{\arabic{section}(\alph{subsection})}
\makeatletter
\newenvironment{pr}{\@startsection % section as pr
       {section}{1}
       {0.4em}{-.5ex plus -1ex minus -.2ex}{.5ex plus .2ex}
       {\pagebreak[3]\large\bf\noindent{Problem}}}
       {\nopagebreak[3]\vspace{3ex}}
\newenvironment{pa}{\@startsection % subsection as pa
       {subsection}{2}
       {0.3em}{0ex plus -1ex minus -.2ex}{.5ex plus .2ex}
       {\pagebreak[3]\large\noindent{}}}
       {\nopagebreak[3]\vspace{3ex}}
\makeatother
\usepackage{fancyhdr}
\pagestyle{fancy}
\lhead{\footnotesize Adrien Fallou} % header left
\chead{\footnotesize} % header center
\rhead{\thepage} % header right
\lfoot{} 
\cfoot{} 
\rfoot{} 
\renewcommand{\headrulewidth}{.3pt} 
\renewcommand{\footrulewidth}{.3pt}
\setlength\voffset{-0.25in}
\setlength\textheight{648pt}

\setlength{\tabcolsep}{8pt}
\renewcommand{\arraystretch}{1.2}
% big-O notation
\newcommand{\bigO}[1]{\ensuremath{\mathop{}\mathopen{}\mathcal{O}\mathopen{}\left(#1\right)}}
%%%%%%%%%%%%%%%%%%%%%%%%%%%%%%%

%%%%%%%%%%%%%%%%%%%%%%%%%%%%%%
% Code blocks formatting
\usepackage{listings}
\usepackage{color}

\definecolor{dkgreen}{rgb}{0,0.6,0}
\definecolor{gray}{rgb}{0.5,0.5,0.5}
\definecolor{mauve}{rgb}{0.58,0,0.82}

\lstset{frame=tb,
  language=matlab,
  aboveskip=3mm,
  belowskip=3mm,
  showstringspaces=false,
  columns=flexible,
  basicstyle={\scriptsize\ttfamily},
  numbers=none,
  numberstyle=\tiny\color{gray},
  keywordstyle=\color{blue},
  commentstyle=\color{dkgreen},
  stringstyle=\color{mauve},
  breaklines=true,
  breakatwhitespace=true,
  tabsize=3
}

\makeatletter
\newenvironment{CenteredBox}{% 
\begin{Sbox}}{% Save the content in a box
\end{Sbox}\centerline{\parbox{\wd\@Sbox}{\TheSbox}}}% And output it centered
\makeatother

%%%%%%%%%%%%%%%%%%%%%%%%%%%%%%
\begin{document}

  \title{CS 229 : Project progress report}
  \author{D.Deriso, N. Banerjee, A. Fallou}
  \date{\today}
  \maketitle
  \thispagestyle{empty}
  %%%%%%%%%%%%%%%%%%%%%%%%%%%%%%%

\large Introduction \\
%
\small Recent advances in mobile computing power have opened up myriad of possibilities in continuous monitoring of health. 
This, combined with the camera faculty available in most smartphones has enabled heart rate detection from mobile phone video. 
We believe yet more information can be gleaned from mobile phone video data. 
Our aim is to predict the pulse oximeter waveform of a person purely from a video of them. 
This would give us both the heart rate and the $O_2$ saturation of the blood without the need for medical equipment.

Why we think this can work will be justified briefly in two steps. 
The first comes from a paper released in 2011 from the MIT Computer Science and Artifical Intelligence Lab. 
They use a procedure called Eulerian Video Magnification (http://people.csail.mit.edu/mrub/vidmag/).  
Most importantly however, it demonstrates that these features are in fact present in the video data.

The second part of the justification is that pulse oximetry as a technique relies on the change in colour of skin. 
Oxygenated and de-oxygenated blood have slightly different hues of red.
This is caused by the difference in hemoglobin levels and this color difference is what the pulse oximeter looks for. 
We intend to look for this change too.

We believe these two ideas imply that there is information present in the video to reconstruct the pulse oximeter waveform.\\
\\
\\
\\ % Left space for Adrien's bit
\\
\large Gathering the data \\
\small For each video, we create a four dimensional matrix of numbers representing (y location of pixel, x location, red, green or blue channel, frame).
For each pixel we have three channels (RGB) and we take the discrete Fourier transform (DFT) of each channel over the whole video.
We then discretize the DFT and sort it into a set number of bins. 
Initially we set the number of bins, n\_bins = 10.
This may well change as the project advances.
Each pixel is considered a different training example and our input feature vector per pixel contains: 
the x,y coordinates of each pixel, and the values of each of the bins per channel (ie 2 + 3*n\_bins features)
giving us a feature vector of size 32.
Our output is a vector of values corresponding to k frequencies/bins for the waveform.
Thus our parameters matrix, $\theta \in \mathbb{R}^{(3*n\_bins + 2) \times k}$

It is worth noting that when choosing our bin sizes and locations, we have assumed a prior on our expected waveform. 
We have chosen to look only at low frequencies (between 0-5 Hz) and 
ignore higher ones because we expect the pulse-ox waveform will contain  mostly low frequencies.
We also expect any high frequencies are more likely to be noise.
This reduces the dimension of our dataset too.\\

\large Initial results \\
\small Our initial computations on the pixel-intensity data revealed, unsuprisingly,
that pixel intensity variation over time for a reasonably still video was not large.
Often a single color channel for a pixel would have a range of 3 (when it could vary from 0-255) over the whole video, even if that pixel was contained in the face.
The DFT of the data would produce large peaks at 0 Hz and would rapidly diminish in magnitude.
We note that we may have to resort to using EVM-enhanced videos if our training data has signals that are too weak to pick up.


  %%%%%%%%%%%%%%%%%%%%%%%%%%%%%%%
\end{document}


